\newcommand{\projectName}{SNail}


\documentclass{sigchi}

% Use this command to override the default ACM copyright statement
% (e.g. for preprints).  Consult the conference website for the
% camera-ready copyright statement.


%% EXAMPLE BEGIN -- HOW TO OVERRIDE THE DEFAULT COPYRIGHT STRIP -- (July 22, 2013 - Paul Baumann)
% \toappear{Permission to make digital or hard copies of all or part of this work for personal or classroom use is      granted without fee provided that copies are not made or distributed for profit or commercial advantage and that copies bear this notice and the full citation on the first page. Copyrights for components of this work owned by others than ACM must be honored. Abstracting with credit is permitted. To copy otherwise, or republish, to post on servers or to redistribute to lists, requires prior specific permission and/or a fee. Request permissions from permissions@acm.org. \\
% {\emph{CHI'14}}, April 26--May 1, 2014, Toronto, Canada. \\
% Copyright \copyright~2014 ACM ISBN/14/04...\$15.00. \\
% DOI string from ACM form confirmation}
%% EXAMPLE END -- HOW TO OVERRIDE THE DEFAULT COPYRIGHT STRIP -- (July 22, 2013 - Paul Baumann)


% Arabic page numbers for submission.  Remove this line to eliminate
% page numbers for the camera ready copy 

%\pagenumbering{arabic}

% Load basic packages
\usepackage{balance}  % to better equalize the last page
\usepackage{graphics} % for EPS, load graphicx instead 
%\usepackage[T1]{fontenc}
\usepackage{txfonts}
\usepackage{times}    % comment if you want LaTeX's default font
\usepackage[pdftex]{hyperref}
% \usepackage{url}      % llt: nicely formatted URLs
\usepackage{color}
\usepackage{textcomp}
\usepackage{booktabs}
\usepackage{ccicons}
\usepackage{todonotes}

% llt: Define a global style for URLs, rather that the default one
\makeatletter
\def\url@leostyle{%
  \@ifundefined{selectfont}{\def\UrlFont{\sf}}{\def\UrlFont{\small\bf\ttfamily}}}
\makeatother
\urlstyle{leo}

% To make various LaTeX processors do the right thing with page size.
\def\pprw{8.5in}
\def\pprh{11in}
\special{papersize=\pprw,\pprh}
\setlength{\paperwidth}{\pprw}
\setlength{\paperheight}{\pprh}
\setlength{\pdfpagewidth}{\pprw}
\setlength{\pdfpageheight}{\pprh}

% Make sure hyperref comes last of your loaded packages, to give it a
% fighting chance of not being over-written, since its job is to
% redefine many LaTeX commands.
\definecolor{linkColor}{RGB}{6,125,233}
\hypersetup{%
  pdftitle={SIGCHI Conference Proceedings Format},
  pdfauthor={LaTeX},
  pdfkeywords={SIGCHI, proceedings, archival format},
  bookmarksnumbered,
  pdfstartview={FitH},
  colorlinks,
  citecolor=black,
  filecolor=black,
  linkcolor=black,
  urlcolor=linkColor,
  breaklinks=true,
}

% create a shortcut to typeset table headings
% \newcommand\tabhead[1]{\small\textbf{#1}}

% End of preamble. Here it comes the document.
\begin{document}

\title{\projectName{}: Sensing the Strains From Fingernail As Always-Available Input}% on Any Surface

\numberofauthors{3}
\author{%
  \alignauthor{1st Author Name\\
    \affaddr{Affiliation}\\
    \affaddr{City, Country}\\
    \email{e-mail address}}\\
  \alignauthor{2nd Author Name\\
    \affaddr{Affiliation}\\
    \affaddr{City, Country}\\
    \email{e-mail address}}\\
  \alignauthor{3rd Author Name\\
    \affaddr{Affiliation}\\
    \affaddr{City, Country}\\
    \email{e-mail address}}\\
}

\maketitle
\begin{abstract}
We present \projectName{}, a nail-mounted device that sense user's fingernail contour and bend when force is applied on surface. By using 3$\times$3 array of 0.2mm strain gauges, \projectName{} is small enough to fit within fingernail, and it is flexible and stretchable. Since the device is always available, it allows user to intuitively use smart TV/devices by simply performing gestures on surfaces around without touching devices. We evaluate this interface in motionless and motion mode. The system can achieved 90\% accuracy for classifying with different kinds of finger posture angle, levels of pressure in motionless mode. For motion mode, it can distinguish 8 directions of movement with high accuracy. We also show some examples applications using this new interaction technique.
\end{abstract}

\keywords{Natural User Interface (NUI); Wearable electronics; fingernail; Strain gauges; Machine Learning; Nail pressure;}

\category{H.5.m.}{Information Interfaces and Presentation
  (e.g. HCI)}{Input devices and strategies (e.g., mouse, touchscreen)} 

\section{Introduction}
%must have this!
In summary, the main contributions of this paper are as follows:
 \begin{itemize} %placeholder for now
 \item Propose the new novel way to input on surface
 \item Propose the new novel way to input on surface
 \item Propose the new novel way to input on surface
 \item Propose the new novel way to input on surface
 \end{itemize}

\section{Related Work}
\subsection{Functional work}
Light ring provide an always-available 2D input on any surface. LightRing uses IR emmitter and gyroscope to acquire finger movements on any surface. IR emmitter is used for measuring the distance between the ring and the middle segment (middle phalanx) of the instrumented finger, and gyroscope is used for rotation rate. However, it need to correct for drift with a magnetometer for other longer interactions.
\subsection{Camera Approches}
NailSense use camera to track the finger tip’s color change, and then guess user is performing the touch or released gestures (using OpenCV). The limitation is camera needed and the gestures need to be performed in front of it. AirPincher is a handheld device which provides eye-free input within fingers and support 6 kinds of gesture with tactile feedback. User can use his/her thumb to pinch, swipe, rubbing at index and middle finger, and the demo of surfing websites also proposed. Nevertheless, camera needed and need to hold the device, not always avaliable.
\subsection{Location work}
MagNail is a study that augmenting nails with a magnet to detect user actions using a smart device which allows user actions to be detected via the magnetic sensor integrated in smart devices such as a smartphone or a tablet PC. The limitation is the error rates is pretty high on the button of the device. NailO use capacitive sensing on printed electrodes as a input surface. It is small and unobtrusive, existence could be forgotten and it has a high accuracy. However, it can onli support 5 kinds of gestures. 
\subsection{Acoustic work}
TapSense use a acoustic way to identify which gesture is pefromed by the users. They capture the sound when our fingers strike the screen, and use that sound pattern to recongnize the gesture. This approach doesn't require user to wear anything, but will need a microphone which already built in smart phones. Scratch input use the sound when fingernail is dragged over the surface of a textured material as finger input surface. Six Scratch Input gestures at about 90\% accuracy with less than five minutes of training and on wide variety of surfaces. However, the interface table will not always exist.

\section{Hardware design}

\subsection{Sensing Touch Angle}
\subsection{Sensing Force Level}
\subsection{Sensing Movement}

\section{Prototype Design}
Physical changes on the surface of nail caused by finger motions and gestures on surface are visible to the human eye. However, there is still difficulty on detecting it through most sensing techniques because such physical changes are very small in comparison to that of finger joint movements. Moreover, physical changes on the surface of nail varies from person to person. Accordingly, we have two major requirements. First, a small and light device must be able to sense small physical changes. Second, a device must be able to extract individual information from multiple spots on the surface of a human nail. Among a variety of sensing techniques, we conclude that strain gauge sensors best fit our requirements for this prototype.
\subsection{Hardware}
Strain gauge sensors are sensitive devices used to measure very small strain and elongation of an object such as buildings, foundations, and other structures. In order to accomplish such measurement, a strain gauge must be adhesively attached to the object. Any tension or compression on the object leads to changes in the electrical resistance of the sensor.
% a figure here

In our prototype, the sensing stage of our prototype consists of a reusable hydrogel-based artificial skin and 9 [120-ohm 2-mm] strain gauge sensors. The artificial skin is attached to the surface of nail and acts as a sensor carrier. As shown in \autoref{fig:tie}, the sensors are distributed into a 3-by-3 grid configuration. The following sentences explain for such sensor layout. An average of human nail width is approximately [], and the base length of a [2-mm strain gauge is 6.3 mm]. In order to avoid physical contact between any two neighboring sensors causing strain interference, a spaced interval of [2 mm] separates them, resulting in 9 strain gauges in a 3-by-3 grid across the nail surface. The sensor array measures the deformations of multiple individual spots of the artificial skin caused by physical changes on the surface of nail. Thus, our prototype is made possible to become a finger gesture interface for general users.

Next, the signal conditioning stage includes a Wheatstone bridge, 2 8-to-1 analog switches (MAX4617, Maxim Integrated), a dual digital potentiometer (MCP42X1, Microchip), and an instrumental amplifier (INA122U, Analog Devices). Finally, the processing stage includes an Arduino Nano used to sample the sensors and control chips in the signal conditioning stage. The complete circuit diagram is shown in \autoref{fig:completeCircuitDiagram} and a customized circuit board is shown in \autoref{fig:hardware}.

The 9 sensors are read sequentially through the analog switches, so each sensor is electrically connected at a time. Once a sensor is active, the sensor became one of the 4 resistors on the Wheatstone bridge. Since the resistance of each sensor varies under no applied external force, a digital potentiometer in series with the connected sensor is used for calibration. Another digital potentiometer serves the purpose of amplifier gain adjustment. In our prototype, the instrumental amplifier gain is set to be approximately 500. Due to the dynamic response time of the amplifier, a [100-microsecond delay] is added between each reading leading to a sampling rate of 75 Hz for our prototype.
\subsection{Software}

\section{Evaluation: Motionless mode}
The goal of this study is to explore whether the system is capable for classifying different kind of finger posture angle. Participants were asked to reproduce a series of instructed angle \cite{UnderstandingTouch} and force between 1N(Newton) to 5N.
\subsection{Participants}
We recruited 16 participants (13 male, 3 female) between the ages of 20 and 23. All participants were right-handed and drew with their right index fingers on the surface. Each participants received \$5 after one hour experiment.
\subsection{Apparatus} %設備!
% TODO add Study one Figs
The apparatus is shown in Fig???. We used the electronic load-cell to measure the force from finger and put it in front of user. The error of the force measurement is plus and minus 0.1 grams. We also put a 9DOF sensor on user's index finger, it is used for checking whether user is performing the right position and angle.

\subsection{Task and Procedure}
In each trail, the participants were instructed to adjust their finger pitch and roll angle which are selected from \cite{UnderstandingTouch} and the forces between 1N to 5N as shown in Fig???. The participants are also asked to straighten finger during all experiment.
In front of the user, there is a screen showing current and instructed angle and force. After each trail, participants 


%In each trial, after the participant indicated his readiness, one of the drawings in Figure 3 appeared on a Keynote slide. The participant was asked an eyes-free input, that is, not to look at his left palm when sketching on it by right index finger; also, the starting point and stroke order of each drawing were provided. The trial was completed when the participant finished the sketch and dropped his right arm to get ready for the next trial.
%During the trial, the participant needed to wear a partial blind- fold modified from a sanitary mask to completely occlude the users view of his hand as shown in Figure 2.
\subsection{Results}


\section{Evaluation: Motion mode}
\subsection{Participants}
\subsection{Apparatus}
\subsection{Task and Procedure}
\subsection{Data Processing}
\subsection{Results}


\section{Interaction design space}

\section{Discussion and Future Work}

\section{Conclusion}

% \subsection{Title and Authors}
% \subsubsection{Sub-subsections}
% $\times$
% \texttt{.cls}

%%%

% Use a numbered list of references at the end of the article, ordered
% alphabetically by first author, and referenced by numbers in
% brackets~\cite{ethics, Klemmer:2002:WSC:503376.503378,
%   Mather:2000:MUT, Zellweger:2001:FAO:504216.504224}. For papers from
% conference proceedings, include the title of the paper and an
% abbreviated name of the conference (e.g., for Interact 2003
% proceedings, use \textit{Proc. Interact 2003}). Do not include the
% location of the conference or the exact date; do include the page
% numbers if available. See the examples of citations at the end of this
% document. Within this template file, use the \texttt{References} style
% for the text of your citation.

% Your references should be published materials accessible to the
% public.  Internal technical reports may be cited only if they are
% easily accessible (i.e., you provide the address for obtaining the
% report within your citation) and may be obtained by any reader for a
% nominal fee.  Proprietary information may not be cited. Private
% communications should be acknowledged in the main text, not referenced
% (e.g., ``[Robertson, personal communication]'').
%
%\begin{table}
%  \centering
%  \begin{tabular}{r c c}
%    \toprule
%    & \multicolumn{2}{c}{\small{\textbf{Caption}}} \\
%    \cmidrule(r){2-3}
%    {\small\textbf{Objects}}
%    & {\small \textit{Pre-2002}}
%    & {\small \textit{Current}} \\
%    \midrule
%    Tables & Above & Below \\
%    Figures & Below & Below \\
%    \bottomrule
%  \end{tabular}
%  \caption{Table captions should be placed below the table. We
%    recommend table lines be 1 point, 25\% black. Minimize use of
%    unnecessary table lines.}~\label{tab:table1}
%\end{table}

%\begin{figure*}
%  \centering
%  \includegraphics[width=2\columnwidth]{figures/map}
%  \caption{In this image, the map maximizes use of space. You can make
%    figures as wide as you need, up to a maximum of the full width of
%    both columns. Note that \LaTeX\ tends to render large figures on a
%    dedicated page. Image: \ccbynd~ayman on
%    Flickr.}~\label{fig:figure2}
%\end{figure*}

% \begin{itemize}
% \item text
% \end{itemize}
% \ref{tab:table1}

% \begin{enumerate}
% \item text
% \end{enumerate}


%\section{Acknowledgments}
%
%Sample text: We thank all the volunteers, and all publications support
%and staff, who wrote and provided helpful comments on previous
%versions of this document. Authors 1, 2, and 3 gratefully acknowledge
%the grant from NSF (\#1234--2012--ABC). \textit{This whole paragraph is
%  just an example.}

% Balancing columns in a ref list is a bit of a pain because you
% either use a hack like flushend or balance, or manually insert
% a column break.  http://www.tex.ac.uk/cgi-bin/texfaq2html?label=balance
% multicols doesn't work because we're already in two-column mode,
% and flushend isn't awesome, so I choose balance.  See this
% for more info: http://cs.brown.edu/system/software/latex/doc/balance.pdf
%
% Note that in a perfect world balance wants to be in the first
% column of the last page.
%
% If balance doesn't work for you, you can remove that and
% hard-code a column break into the bbl file right before you
% submit:
%
% http://stackoverflow.com/questions/2149854/how-to-manually-equalize-columns-
% in-an-ieee-paper-if-using-bibtex
%
% Or, just remove \balance and give up on balancing the last page.
%
\balance{}


% REFERENCES FORMAT
% References must be the same font size as other body text.
\bibliographystyle{SIGCHI-Reference-Format}
\bibliography{sample}

\end{document}

%%% Local Variables:
%%% mode: latex
%%% TeX-master: t
%%% End:
